\chapter{Discussion, conclusion and future work}
    \begin{enumerate}
        \item We have shown that it is natural and beneficial to apply topological regularizations in some domains, if one has meta information or wants to impose inductive bias on structures that occur.
        \item Author believes that proposed regularization for point cloud regression/optimization may be quite useful if equipped within correct setup and it's best performance regimes are yet to be found/tuned.
        \item Employment of higher dimensions of homologies still remains somewhat a challenge, since one needs a parallel algorithm with feasible computation time and memory requirements.
        \item Proposed methods suffer to a degree from curse of dimensionality and it is an area of possible improvement to test different dimensionality reduction techniques and feature extractors.
        \item There are many examples of structured point clouds and data in the field, and consequently much more applications of similar losses and metrics.
        \item Although not certain, we suspect MTop-Div${}_0$ optimization to be connected with Optimal Transport. Such connection may shed light on how exactly one should balance loss terms in the presence of this regularizer, but also unveil, 'topological' nature of OT, which is, on the surface, very much a metric-dependent method.
    \end{enumerate}
